\section{Physical Interpretation of the Curvature of the Gradient Richardson Number}

The curvature of the gradient Richardson number,
\begin{equation}
\frac{\partial^2 Ri_g}{\partial z^2},
\end{equation}
quantifies how rapidly nonlinear stability effects develop with height.
Physically, it represents the acceleration or deceleration of stratification effects
as turbulent transport becomes increasingly suppressed or enhanced.
Near the surface, this curvature is governed primarily by the shape of the stability
correction functions, $\phi_m$ and $\phi_h$.

In nondimensional coordinates, using the stability variable $\zeta = z/L$,
the relevant dimensionless curvature is
\begin{equation}
\frac{\partial^2 Ri_g}{\partial \zeta^2},
\end{equation}
which depends on $\zeta$ and the stability parameters
$\alpha_m, \beta_m, \alpha_h,$ and $\beta_h$.

% Add a brief note on chain rule (z vs ζ)
\paragraph{Note (z vs.\ $\zeta$).}
Our target is the curvature with respect to height, $\partial^2 Ri_g/\partial z^2$.
We first derive in the nondimensional coordinate $\zeta=z/L$ for compactness, then use the chain rule
$\partial/\partial z = (1/L)\,\partial/\partial \zeta$ (treating $L$ locally constant), hence
$\partial^2/\partial z^2=(1/L^2)\,\partial^2/\partial \zeta^2$.

%-----------------------------------------------------------
\subsection{Logarithmic Derivatives and Sensitivity Functions}

To connect directly with Monin--Obukhov similarity theory (MOST),
the curvature can be expressed in terms of the
logarithmic derivatives of the stability functions:
\begin{equation}
V_{\text{log}} = \frac{1}{\phi_m}\frac{d\phi_m}{d\zeta}, \qquad
H_{\text{log}} = \frac{1}{\phi_h}\frac{d\phi_h}{d\zeta}.
\end{equation}
These quantities describe the sensitivity of the stability corrections
to stratification and are more physically meaningful than their absolute derivatives.

For the power-law forms,
\begin{equation}
\phi(\zeta) = (1 - \beta\zeta)^{-\alpha},
\end{equation}
the logarithmic derivative becomes
\begin{equation}
\frac{1}{\phi}\frac{d\phi}{d\zeta}
= \frac{\alpha\beta}{1 - \beta\zeta}.
\end{equation}

%-----------------------------------------------------------
\subsection{Decomposition of the Curvature}

The curvature of $Ri_g$ may be decomposed into three physically
interpretable contributions:

\paragraph{Term A — Linear Sensitivity.}
\begin{equation}
\textbf{A:} \quad \frac{\partial F}{\partial \zeta},
\end{equation}
which represents the initial rate of change of the ratio
$F = \phi_h / \phi_m^2$,
indicating how quickly the turbulent Prandtl number departs from its neutral value.

\paragraph{Term B — Nonlinear Amplification.}
\begin{equation}
\textbf{B:} \quad
\left( \frac{\partial F}{\partial \zeta} \right)^2,
\end{equation}
arising from the square of the logarithmic derivative of $F(\zeta)$, scaled by $\zeta$,
and describing nonlinear amplification of curvature under increasing stratification.

\paragraph{Term C — Intrinsic Curvature.}
\begin{equation}
\textbf{C:} \quad \frac{\partial^2 F}{\partial \zeta^2},
\end{equation}
which isolates the curvature intrinsic to the stability functions themselves.
For the power-law form,
\begin{equation}
\frac{d}{d\zeta}\left(\frac{1}{\phi}\frac{d\phi}{d\zeta}\right)
= \frac{\alpha\beta^2}{(1 - \beta\zeta)^2}.
\end{equation}

%-----------------------------------------------------------
\subsection{Compact Analytical Expression}

Collecting all terms, the dimensionless curvature may be expressed as
\begin{equation}
\frac{\partial^2 Ri_g}{\partial \zeta^2}
\propto
F(\zeta)
\left[
C_1\,V_{\text{log}}
+ C_2\,\zeta\,V_{\text{log}}^2
+ C_3\,W_{\text{log}}
\right],
\label{eq:rig_curvature}
\end{equation}
where each coefficient $C_i$ depends on the power-law parameters
$(\alpha,\beta)$ for heat and momentum,
and
\begin{equation}
W_{\text{log}} =
\frac{dV_{\text{log}}}{d\zeta}.
\end{equation}
Because $F(\zeta)$ and its derivatives are rational functions of
$(1 - \beta_m\zeta)$ and $(1 - \beta_h\zeta)$,
the singularities at $\zeta = 1/\beta_m" and "\zeta = 1/\beta_h"
naturally appear, corresponding to the asymptotic limits of strong stability or instability.

%-----------------------------------------------------------
\subsection{Neutral Limit $(\zeta \to 0)$}

Near neutrality, where $\phi_m, \phi_h \to 1$, the curvature simplifies to
\begin{equation}
F(0) = 1, \qquad
V_{\text{log}}(0) = \alpha_h\beta_h - 2\alpha_m\beta_m, \qquad
W_{\text{log}}(0) = \alpha_h\beta_h^2 - 2\alpha_m\beta_m^2.
\end{equation}
Thus, at the surface ($\zeta = 0$),
\begin{equation}
\left.\frac{\partial^2 Ri_g}{\partial \zeta^2}\right|_{\zeta=0}
\propto
(\alpha_h\beta_h - 2\alpha_m\beta_m)
+ (\alpha_h\beta_h^2 - 2\alpha_m\beta_m^2).
\label{eq:neutral_curvature}
\end{equation}

%-----------------------------------------------------------
\subsection{Physical Interpretation}

\begin{itemize}
  \item If $\alpha_h\beta_h \approx 2\alpha_m\beta_m$:
  the curvature vanishes, and $Ri_g$ grows \emph{linearly} with $\zeta$;
  heat and momentum corrections evolve similarly.

  \item If $\alpha_h\beta_h < 2\alpha_m\beta_m$:
  $\partial^2 Ri_g / \partial \zeta^2 < 0$,
  and the $Ri_g$ profile is \emph{concave down} near the surface,
  indicating rapid strengthening of stratification with height.

  \item If $\alpha_h\beta_h > 2\alpha_m\beta_m$:
  $\partial^2 Ri_g / \partial \zeta^2 > 0$,
  and the $Ri_g$ profile is \emph{concave up},
  indicating that stratification initially increases more slowly with height.
\end{itemize}

%-----------------------------------------------------------
\subsection{Discussion and Implications}

The curvature of $Ri_g(\zeta)$ provides a compact yet powerful diagnostic
of the nonlinear coupling between buoyancy and shear in the surface layer.
While $Ri_g$ itself is often treated as a monotonic function of stability,
its curvature reveals the \emph{rate} at which turbulent exchange efficiency
responds to stratification.

In practical terms, the sign and magnitude of
$\partial^2 Ri_g / \partial \zeta^2$
govern how quickly the flow transitions between turbulence-dominated and
stratification-dominated regimes.
A negative curvature (concave-down $Ri_g$) indicates that
the suppression of turbulence by buoyancy strengthens rapidly with height,
producing a shallow surface layer and an early collapse of momentum flux.
Conversely, a positive curvature (concave-up $Ri_g$)
implies that turbulent transport remains effective over a deeper layer,
delaying the onset of laminarization.

The neutral-limit form [Eq.~(\ref{eq:neutral_curvature})]
demonstrates that this curvature depends only on the linear coefficients
$\alpha\beta$ of the stability functions---parameters that characterize
the first-order departure from neutrality.
Hence, the curvature serves as a sensitive indicator of the
\emph{relative efficiency} of heat and momentum transfer under weakly
stratified conditions.

This interpretation aligns with large-eddy simulation (LES) and field observations
that report distinct curvature regimes:
stable cases typically exhibit a rapid (negative) curvature,
while near-neutral or weakly unstable flows maintain nearly linear behavior
of $Ri_g$ with height.
Under strong instability, the curvature may again become positive,
reflecting enhanced mixing and the deepening of the convective boundary layer.

From a modeling standpoint, this analytical structure provides a
useful constraint for parameterization schemes.
By requiring that $\partial^2 Ri_g / \partial \zeta^2$ remain finite and physically
consistent across stability regimes,
one can ensure that the parameterized $\phi_m$ and $\phi_h$ functions
yield smooth transitions between turbulent and weakly stable layers.
In this sense, the curvature analysis bridges the formal structure of
Monin--Obukhov similarity with the empirically observed variability
of surface-layer turbulence.

%-----------------------------------------------------------
\subsection*{Chain rule to height coordinate}
Using $\zeta=z/L$ with locally constant $L$ at the evaluation level,
\begin{equation}
\boxed{\;\frac{\partial^2 Ri_g}{\partial z^2}
\;=\; \frac{1}{L^2}\,\frac{\partial^2 Ri_g}{\partial \zeta^2}\;}
\end{equation}
If vertical variation of $L$ is retained, additional terms proportional to $dL/dz$ appear; in MOST surface-layer usage we evaluate with the local $L$ and use the factor $1/L^2$.

%-----------------------------------------------------------
\subsection*{Enhanced MOST / Richardson Number Formulations}

Next-generation atmospheric and climate models benefit from stability functions that (i) avoid hard singularities, (ii) embed dynamic turbulent Prandtl behavior, (iii) couple to nonlocal mixing under convective or very stable regimes, and (iv) permit smooth Ri-based inversion.

\paragraph{1. Regularized Power-Law (RPL).}
Replace $(1-\beta\zeta)^{-\alpha}$ by
\[
\phi^{\text{RPL}}(\zeta)=\bigl(1+\gamma(\beta\zeta)\bigr)^{\alpha},\qquad
\gamma(x)=\frac{x}{1+\delta x},
\]
which removes the finite-height pole; choose $\delta>0$ small (e.g.\ $0.05$) for near-neutral fidelity.

\paragraph{2. Variable-Exponent Form (VEXP).}
Allow weak height variation:
\[
\phi_m=(1-\beta_m\zeta)^{-\alpha_m(1+\eta_m\zeta)},\quad
\phi_h=(1-\beta_h\zeta)^{-\alpha_h(1+\eta_h\zeta)}.
\]
Neutral curvature calibration gives $\eta_{m,h}$; preserves analytic derivatives for curvature and Ri inversion.

\paragraph{3. Ri-Conditioned Blend (RB).}
Define bulk/gradient Richardson proxy $Ri^\star(\zeta)=\zeta \phi_h/\phi_m^2$ and blend MOST with asymptotic shear-limited form:
\[
\phi_m^{\text{RB}}=\phi_m \Bigl[1-\chi(Ri^\star)\Bigr] + \phi_m^{\infty}\chi(Ri^\star),
\]
with $\chi(Ri)=\frac{Ri^p}{Ri^p+R_c^p}$; $p\sim 2$, $\phi_m^{\infty}$ a linear shear profile factor.

\paragraph{4. Dynamic Turbulent Prandtl (DTP).}
Specify
\[
Pr_t(\zeta)=\frac{\phi_h}{\phi_m} = 1 + a_1 Ri^\star + a_2 (Ri^\star)^2,
\]
then solve for modified $\phi_h = Pr_t \phi_m$ ensuring neutral $Pr_t\to1$ and stable asymptote $Pr_t\to Pr_t^{\max}$.

\paragraph{5. Nonlocal Augmentation (NLM).}
For convective $-\zeta>0$ or shear-generated intermittent stable layers:
\[
\phi_m^{\text{NLM}}=\phi_m \Bigl(1 + c_m \frac{z}{h_{mix}}\Bigr),\quad
\phi_h^{\text{NLM}}=\phi_h \Bigl(1 + c_h \frac{z}{h_{mix}}\Bigr),
\]
with prognosed mixing depth $h_{mix}$; maintains correct near-surface limit and adds linear nonlocal transport.

\paragraph{6. Curvature Guard / Soft Limiter.}
Apply
\[
\phi_{m,h}^{\text{guard}}=\phi_{m,h}\Big/\Bigl[1 + \epsilon_{m,h}\, \Bigl|\zeta^2 \frac{d^2 Ri_g}{d\zeta^2}\Bigr|\Bigr],
\]
to suppress spurious grid-induced curvature spikes when $\Delta x$ or $\Delta z$ coarse.

\paragraph{7. Unified Ri-Based Closure (URC).}
Given neutral coefficients $\Delta=\alpha_h\beta_h-2\alpha_m\beta_m$, define:
\[
f_m(Ri)=\left(1 + b_m \frac{Ri}{Ri_c}\right)^{-e_m},\quad
e_m=\frac{\alpha_m}{2\alpha_m-\alpha_h},
\]
with analogous $f_h$; match $b_{m,h}$ via neutral curvature and first inflection height if present.

\paragraph{Implementation Pseudocode.}
\begin{verbatim}
def phi_m(zeta, params):
    model = params.model  # 'RPL','VEXP','RB','NLM'
    if model=='RPL':
        beta, alpha, delta = params.beta_m, params.alpha_m, params.delta
        g = (beta*zeta)/(1+delta*beta*zeta)
        return (1+g)**alpha
    elif model=='VEXP':
        beta, alpha, eta = params.beta_m, params.alpha_m, params.eta_m
        return (1 - beta*zeta)**(-alpha*(1+eta*zeta))
    # ...additional branches...
\end{verbatim}

\paragraph{Calibration Strategy.}
\begin{itemize}\setlength\itemsep{2pt}
\item Match neutral curvature $2(\alpha_h\beta_h-2\alpha_m\beta_m)$.
\item Fit $Pr_t$ slope $(\alpha_h\beta_h-\alpha_m\beta_m)$.
\item Constrain large-$Ri$ asymptote to observed $Ri_{crit}$ distribution.
\item Enforce smooth $\partial_\zeta^2 Ri_g$ (no artificial extrema).
\end{itemize}

\paragraph{Benefits.}
Improved numerical stability (removed poles), tunable curvature to reduce early flux collapse, explicit Ri-based mapping eliminating iterative ζ search, and nonlocal augmentation consistent with LES.

%-----------------------------------------------------------
\subsection*{References}
\begin{thebibliography}{99}

\bibitem[Monin and Obukhov(1954)]{MoninObukhov1954}
Monin, A. S., and A. M. Obukhov, 1954: Basic laws of turbulent mixing in the ground layer of the atmosphere. Trudy Geofiz. Inst. AN SSSR, 151, 163–187.

\bibitem[Businger et al.(1971)]{Businger1971}
Businger, J. A., J. C. Wyngaard, Y. Izumi, and E. F. Bradley, 1971: Flux–profile relationships in the atmospheric surface layer. J. Atmos. Sci., 28, 181–189.

\bibitem[Paulson(1970)]{Paulson1970}
Paulson, C. A., 1970: The mathematical representation of wind speed and temperature profiles in the unstable atmospheric surface layer. J. Appl. Meteor., 9, 857–861.

\bibitem[Dyer(1974)]{Dyer1974}
Dyer, A. J., 1974: A review of flux–profile relationships. Boundary-Layer Meteorol., 7, 363–372.

\bibitem[Högström(1988)]{Hogstrom1988}
Högström, U., 1988: Non-dimensional wind and temperature profiles in the atmospheric surface layer—A re-evaluation. Boundary-Layer Meteorol., 42, 55–78.

\bibitem[Garratt(1992)]{Garratt1992}
Garratt, J. R., 1992: The Atmospheric Boundary Layer. Cambridge University Press, 316 pp.

\bibitem[Kaimal and Finnigan(1994)]{KaimalFinnigan1994}
Kaimal, J. C., and J. J. Finnigan, 1994: Atmospheric Boundary Layer Flows. Oxford University Press, 289 pp.

\bibitem[Stull(1988)]{Stull1988}
Stull, R. B., 1988: An Introduction to Boundary Layer Meteorology. Kluwer Academic, 670 pp.

\bibitem[Li et al.(2012)]{LiKatulBouZeid2012}
Li, D., G. G. Katul, and E. Bou-Zeid, 2012: Mean velocity and temperature profiles in a sheared diabatic turbulent boundary layer. Phys. Fluids, 24, 105105.

\bibitem[Gryanik et al.(2020)]{Gryanik2020}
Gryanik, V. M., C. Lüpkes, A. Grachev, and D. Sidorenko, 2020: Modified stability functions for the stable boundary layer based on SHEBA. J. Atmos. Sci., 77, 2687–2716.

\bibitem[Mahrt(2014)]{Mahrt2014}
Mahrt, L., 2014: Stably stratified atmospheric boundary layers. Annu. Rev. Fluid Mech., 46, 23–45.

\bibitem[England and McNider(1995)]{EnglandMcNider1995}
England, D. E., and R. T. McNider, 1995: Stability functions based upon shear functions. Boundary-Layer Meteorol., 72, 115–140.

\end{thebibliography}