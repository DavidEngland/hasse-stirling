\section*{Curvature of the Gradient Richardson Number ($Ri_g$) in Power-Law MOST}
\label{sec:rig_curvature}

We analyze the curvature of the gradient Richardson number $Ri_g$ with respect to the dimensionless height $\zeta=z/L$ for power-law Monin-Obukhov Similarity Theory (MOST) stability functions, following the approach of \citep{McNider:YYYY} and \citep{Biazar:YYYY}.

The stability functions for momentum and heat are
$$
\phi_m(\zeta)=(1-\beta_m\zeta)^{-\alpha_m},\qquad \phi_h(\zeta)=(1-\beta_h\zeta)^{-\alpha_h},
\quad (1-\beta_{(\cdot)}\zeta)>0.
$$
The MOST form of the gradient Richardson number is $Ri_g(\zeta)=\zeta F(\zeta)$, where $F(\zeta)=\phi_h(\zeta)/\phi_m(\zeta)^2$.

\subsection*{Curvature Derivation}

We define the logarithmic derivative components:
\begin{align*}
V_{\log}(\zeta) &\equiv \frac{1}{F}\frac{dF}{d\zeta} = \frac{\alpha_h\beta_h}{1-\beta_h\zeta}-\frac{2\alpha_m\beta_m}{1-\beta_m\zeta},\\
W_{\log}(\zeta) &\equiv \frac{dV_{\log}}{d\zeta} = \frac{\alpha_h\beta_h^{2}}{(1-\beta_h\zeta)^2}-\frac{2\alpha_m\beta_m^{2}}{(1-\beta_m\zeta)^2}.
\end{align*}
Using the chain rule identities $\frac{dRi_g}{d\zeta}=F+\zeta F'$ and $\frac{d^{2}Ri_g}{d\zeta^{2}}=2F'+\zeta F''$, along with $F''/F=V_{\log}^2-W_{\log}$, we obtain the fundamental closed expression for the curvature:
\begin{equation}
\label{eq:rig_curvature_compact}
\boxed{\frac{d^{2}Ri_g}{d\zeta^{2}}=F(\zeta)\Big[\,2V_{\log}(\zeta)+\zeta\big(V_{\log}(\zeta)^2-W_{\log}(\zeta)\big)\Big].}
\end{equation}

\subsection*{Neutral Limit and Interpretation}

In the neutral limit ($\zeta\to 0$), $F(0)=1$, $V_{\log}(0)=\Delta$, and $W_{\log}(0)=c_1$, where we define the neutral coefficients:
$$
\Delta\equiv\alpha_h\beta_h-2\alpha_m\beta_m,\qquad c_1\equiv\alpha_h\beta_h^{2}-2\alpha_m\beta_m^{2}.
$$
The curvature at neutrality is then:
\begin{equation}
\label{eq:rig_curvature_neutral}
\boxed{\left.\frac{d^{2}Ri_g}{d\zeta^{2}}\right|_{\zeta=0}=2\Delta.}
\end{equation}
The sign of $\Delta$ determines the initial concavity of $Ri_g(\zeta)$: $\Delta>0$ implies concave-up (heat corrections dominate), and $\Delta<0$ implies concave-down (momentum corrections dominate). The leading-order behavior is $Ri_g(\zeta)=\zeta+\Delta\zeta^2+O(\zeta^3)$.

\subsection*{Physical Height Conversion}

The curvature with respect to the physical height $z$ is related to the $\zeta$-curvature via the local-L chain rule:
\begin{equation}
\label{eq:rig_curvature_physical}
\boxed{\frac{\partial^2 Ri_g}{\partial z^2}=\frac{1}{L^2}\frac{d^{2}Ri_g}{d\zeta^{2}}.}
\end{equation}
\end{itemize}